\documentclass[a4paper,titlepage,12pt]{article}\usepackage[]{graphicx}\usepackage[]{color}
%% maxwidth is the original width if it is less than linewidth
%% otherwise use linewidth (to make sure the graphics do not exceed the margin)
\makeatletter
\def\maxwidth{ %
  \ifdim\Gin@nat@width>\linewidth
    \linewidth
  \else
    \Gin@nat@width
  \fi
}
\makeatother

\definecolor{fgcolor}{rgb}{0.345, 0.345, 0.345}
\newcommand{\hlnum}[1]{\textcolor[rgb]{0.686,0.059,0.569}{#1}}%
\newcommand{\hlstr}[1]{\textcolor[rgb]{0.192,0.494,0.8}{#1}}%
\newcommand{\hlcom}[1]{\textcolor[rgb]{0.678,0.584,0.686}{\textit{#1}}}%
\newcommand{\hlopt}[1]{\textcolor[rgb]{0,0,0}{#1}}%
\newcommand{\hlstd}[1]{\textcolor[rgb]{0.345,0.345,0.345}{#1}}%
\newcommand{\hlkwa}[1]{\textcolor[rgb]{0.161,0.373,0.58}{\textbf{#1}}}%
\newcommand{\hlkwb}[1]{\textcolor[rgb]{0.69,0.353,0.396}{#1}}%
\newcommand{\hlkwc}[1]{\textcolor[rgb]{0.333,0.667,0.333}{#1}}%
\newcommand{\hlkwd}[1]{\textcolor[rgb]{0.737,0.353,0.396}{\textbf{#1}}}%

\usepackage{framed}
\makeatletter
\newenvironment{kframe}{%
 \def\at@end@of@kframe{}%
 \ifinner\ifhmode%
  \def\at@end@of@kframe{\end{minipage}}%
  \begin{minipage}{\columnwidth}%
 \fi\fi%
 \def\FrameCommand##1{\hskip\@totalleftmargin \hskip-\fboxsep
 \colorbox{shadecolor}{##1}\hskip-\fboxsep
     % There is no \\@totalrightmargin, so:
     \hskip-\linewidth \hskip-\@totalleftmargin \hskip\columnwidth}%
 \MakeFramed {\advance\hsize-\width
   \@totalleftmargin\z@ \linewidth\hsize
   \@setminipage}}%
 {\par\unskip\endMakeFramed%
 \at@end@of@kframe}
\makeatother

\definecolor{shadecolor}{rgb}{.97, .97, .97}
\definecolor{messagecolor}{rgb}{0, 0, 0}
\definecolor{warningcolor}{rgb}{1, 0, 1}
\definecolor{errorcolor}{rgb}{1, 0, 0}
\newenvironment{knitrout}{}{} % an empty environment to be redefined in TeX

\usepackage{alltt}
\usepackage[a4paper,top=2cm,bottom=2cm,left=2cm,right=2cm]{geometry}
\usepackage{longtable}
%\usepackage[catalan]{babel}
\usepackage[spanish]{babel}
\usepackage[latin1]{inputenc}
%\usepackage[ansinew]{inputenc}
\usepackage{hyperref}
%\usepackage[pdftex]{color,graphicx,epsfig}
\DeclareGraphicsRule{.pdftex}{pdf}{.pdftex}{}
\usepackage{amssymb,amsmath}
\usepackage{multirow}
\usepackage{lscape}
\usepackage{Sweave}
\usepackage{lscape}
\usepackage{float}
%\usepackage{color} 
\usepackage[latin1]{inputenc}
\usepackage{sectsty}
\usepackage[final]{pdfpages}
\sectionfont{\large}
\usepackage{listings}

%\bffont{\large}
% to change margings in itemize
\newenvironment{itemize*}%
{\begin{itemize}%
 \setlength{\itemsep}{-0.35cm}%
 \setlength{\parskip}{10pt}}%
{\end{itemize}}


% to change margin
\newenvironment{changemargin1}{
  \begin{list}{}{
    \setlength{\leftmargin}{-2cm}
    \setlength{\rightmargin}{3cm}
    \footnotesize
  }
  \item[]
}{\end{list}}



% to make fancy header
\usepackage{fancyhdr}
\pagestyle{fancy}
\fancyfoot{}
\fancyhead[L]{Multivariate Analysis}
\fancyhead[R]{Judith Pe?afiel \qquad \thepage}
\renewcommand{\headrulewidth}{0.5pt} 
\addtolength{\headheight}{5pt}
\IfFileExists{upquote.sty}{\usepackage{upquote}}{}


\begin{document}


\renewcommand{\tablename}{Tabla}
\renewcommand{\listtablename}{\'{I}ndice de tablas} 
\renewcommand{\listfigurename}{\'{I}ndice de figuras}


\title{\bf Exercises Multivariate Analysis}

\vspace{1cm}

\author{Judith Pe?afiel}


\maketitle
\newpage


\section{An?lisi discriminant}
\subsection{Find the mean vectors and the covariance matrices.}
Comen?em llegint les dades
\begin{knitrout}
\definecolor{shadecolor}{rgb}{0.969, 0.969, 0.969}\color{fgcolor}\begin{kframe}
\begin{alltt}
\hlkwd{rm}\hlstd{(}\hlkwc{list} \hlstd{=} \hlkwd{ls}\hlstd{())}
\hlkwd{library}\hlstd{(xtable)}
\hlcom{## setwd('C:/Documents and Settings/jpenafiel/Mis}
\hlcom{## documentos/Dropbox/rutines/')}
\hlkwd{setwd}\hlstd{(}\hlstr{"C:/programs/Dropbox/rutines"}\hlstd{)}
\hlstd{dat} \hlkwb{<-} \hlkwd{read.table}\hlstd{(}\hlstr{"Copepodes.csv"}\hlstd{,} \hlkwc{header} \hlstd{=} \hlnum{TRUE}\hlstd{,} \hlkwc{sep} \hlstd{=} \hlstr{";"}\hlstd{)}
\hlstd{dat} \hlkwb{<-} \hlstd{dat[}\hlopt{-}\hlkwd{nrow}\hlstd{(dat), ]}
\end{alltt}
\end{kframe}
\end{knitrout}

Calculem les mitjanes del vector: 
\begin{itemize}
\item $\tilde{\mu}$ de tota la poblaci?.
\item $\tilde{\mu_{1}}$   del estadi 1.
\item $\tilde{\mu_{2}}$  del estadi 2.
\end{itemize}

\begin{knitrout}
\definecolor{shadecolor}{rgb}{0.969, 0.969, 0.969}\color{fgcolor}\begin{kframe}
\begin{alltt}
\hlstd{xbar0} \hlkwb{<-} \hlkwd{apply}\hlstd{(dat[,} \hlnum{1}\hlopt{:}\hlnum{2}\hlstd{],} \hlnum{2}\hlstd{, mean,} \hlkwc{na.rm} \hlstd{= T)}  \hlcom{## sample mean vector of the all estadi}
\hlstd{xbar1} \hlkwb{<-} \hlkwd{apply}\hlstd{(}\hlkwd{subset}\hlstd{(dat, estadi} \hlopt{==} \hlnum{1}\hlstd{)[,} \hlnum{1}\hlopt{:}\hlnum{2}\hlstd{],} \hlnum{2}\hlstd{, mean)}  \hlcom{## sample mean vector of the first estadi}
\hlstd{xbar2} \hlkwb{<-} \hlkwd{apply}\hlstd{(}\hlkwd{subset}\hlstd{(dat, estadi} \hlopt{==} \hlnum{2}\hlstd{)[,} \hlnum{1}\hlopt{:}\hlnum{2}\hlstd{],} \hlnum{2}\hlstd{, mean)}  \hlcom{## sample mean vector of the second estadi}
\end{alltt}
\end{kframe}
\end{knitrout}

Obtenim com a resultat:
% latex table generated in R 3.0.1 by xtable 1.7-1 package
% Wed Jun 25 14:26:41 2014
\begin{table}[ht]
\centering
\caption{Vector de mitjanes} 
\begin{tabular}{rrr}
  \hline
 & long & amp \\ 
  \hline
Tota la poblaci? & 231.56 & 143.41 \\ 
  Estadi 1 & 219.49 & 138.08 \\ 
  Estadi 2 & 241.64 & 147.86 \\ 
   \hline
\end{tabular}
\end{table}


Calculem les matrius de covari?ncia: 
\begin{knitrout}
\definecolor{shadecolor}{rgb}{0.969, 0.969, 0.969}\color{fgcolor}\begin{kframe}
\begin{alltt}
\hlstd{cov.mat0} \hlkwb{<-} \hlkwd{cov}\hlstd{(dat[,} \hlnum{1}\hlopt{:}\hlnum{2}\hlstd{],} \hlkwc{use} \hlstd{=} \hlstr{"complete.obs"}\hlstd{)}  \hlcom{## covariance matrix of the all estadi}
\hlstd{cov.mat1} \hlkwb{<-} \hlkwd{cov}\hlstd{(}\hlkwd{subset}\hlstd{(dat, estadi} \hlopt{==} \hlnum{1}\hlstd{)[,} \hlnum{1}\hlopt{:}\hlnum{2}\hlstd{])}  \hlcom{## covariance matrix of the first estadi}
\hlstd{cov.mat2} \hlkwb{<-} \hlkwd{cov}\hlstd{(}\hlkwd{subset}\hlstd{(dat, estadi} \hlopt{==} \hlnum{2}\hlstd{)[,} \hlnum{1}\hlopt{:}\hlnum{2}\hlstd{])}  \hlcom{## covariance matrix of the second estadi}
\end{alltt}
\end{kframe}
\end{knitrout}

\begin{itemize}
\item $\tilde{\Sigma}$   de tota la poblaci?.
\begin{knitrout}
\definecolor{shadecolor}{rgb}{0.969, 0.969, 0.969}\color{fgcolor}\begin{kframe}
\begin{verbatim}
##        long    amp
## long 421.91  84.86
## amp   84.86 245.04
\end{verbatim}
\end{kframe}
\end{knitrout}

\newpage
\item$\tilde{\Sigma_1}$  del estadi 1.
\begin{knitrout}
\definecolor{shadecolor}{rgb}{0.969, 0.969, 0.969}\color{fgcolor}\begin{kframe}
\begin{verbatim}
##         long     amp
## long 409.930  -1.316
## amp   -1.316 306.194
\end{verbatim}
\end{kframe}
\end{knitrout}


\item $\tilde{\Sigma_2}$  del estadi 2.
\begin{knitrout}
\definecolor{shadecolor}{rgb}{0.969, 0.969, 0.969}\color{fgcolor}\begin{kframe}
\begin{verbatim}
##         long     amp
## long 409.930  -1.316
## amp   -1.316 306.194
\end{verbatim}
\end{kframe}
\end{knitrout}

\end{itemize}

\subsection{Perform a multivariate comparison of mean groups (In this case you case use an R specific function)}

Utilitzem la proba Hotelling's T$^{2}$ que tasta la hip?tesi :
\\
H$_{0}:\mu=\mu_{0}$
\\
H$_{1}:\mu\neq\mu_{0}$
\\
La f?rmula utilitzada en la seg?ent funci? ?s:
\\
T$^{2}$=$\sqrt{n}(\hat{X}-\mu_{0})S_{pooled}^{-1}\sqrt{n}(\hat{X}-\mu_{0}$)
\\
On,
\\
T$^{2} \sim\frac{(n-1)p}{(n-p)}F_{p,n-p}$
\begin{knitrout}
\definecolor{shadecolor}{rgb}{0.969, 0.969, 0.969}\color{fgcolor}\begin{kframe}
\begin{alltt}
\hlcom{# Funci? test de hotelling}
\hlstd{hotel2T2} \hlkwb{=} \hlkwa{function}\hlstd{(}\hlkwc{x1}\hlstd{,} \hlkwc{x2}\hlstd{,} \hlkwc{a} \hlstd{=} \hlnum{0.05}\hlstd{) \{}
    \hlstd{p} \hlkwb{=} \hlkwd{ncol}\hlstd{(x1)}  \hlcom{## dimenisonality of the data}
    \hlstd{n1} \hlkwb{=} \hlkwd{nrow}\hlstd{(x1)}  \hlcom{## size of the first sample}
    \hlstd{n2} \hlkwb{=} \hlkwd{nrow}\hlstd{(x2)}  \hlcom{## size of the second sample}
    \hlstd{n} \hlkwb{=} \hlstd{n1} \hlopt{+} \hlstd{n2}  \hlcom{## total sample size}
    \hlstd{xbar1} \hlkwb{=} \hlkwd{apply}\hlstd{(x1,} \hlnum{2}\hlstd{, mean)}  \hlcom{## sample mean vector of the first sample}
    \hlstd{xbar2} \hlkwb{=} \hlkwd{apply}\hlstd{(x2,} \hlnum{2}\hlstd{, mean)}  \hlcom{## sample mean vector of the second sample}
    \hlstd{dbar} \hlkwb{=} \hlstd{xbar1} \hlopt{-} \hlstd{xbar2}  \hlcom{## difference of the two mean vectors}
    \hlstd{v} \hlkwb{=} \hlstd{((n1} \hlopt{-} \hlnum{1}\hlstd{)} \hlopt{*} \hlkwd{var}\hlstd{(x1)} \hlopt{+} \hlstd{(n2} \hlopt{-} \hlnum{1}\hlstd{)} \hlopt{*} \hlkwd{var}\hlstd{(x2))}\hlopt{/}\hlstd{(n} \hlopt{-} \hlnum{2}\hlstd{)}  \hlcom{## pooled covariance matrix}
    \hlstd{t2} \hlkwb{=} \hlstd{(n1} \hlopt{*} \hlstd{n2} \hlopt{*} \hlstd{dbar} \hlopt \hlkwd{solve}\hlstd{(v)} \hlopt \hlstd{dbar)}\hlopt{/}\hlstd{n}
    \hlstd{test} \hlkwb{=} \hlstd{((n} \hlopt{-} \hlstd{p} \hlopt{-} \hlnum{1}\hlstd{)} \hlopt{*} \hlstd{t2)}\hlopt{/}\hlstd{((n} \hlopt{-} \hlnum{2}\hlstd{)} \hlopt{*} \hlstd{p)}  \hlcom{## test statistic}
    \hlstd{crit} \hlkwb{=} \hlkwd{qf}\hlstd{(}\hlnum{1} \hlopt{-} \hlstd{a, p, n} \hlopt{-} \hlstd{p} \hlopt{-} \hlnum{1}\hlstd{)}  \hlcom{## critical value of the F distribution}
    \hlstd{pvalue} \hlkwb{=} \hlnum{1} \hlopt{-} \hlkwd{pf}\hlstd{(test, p, n} \hlopt{-} \hlstd{p} \hlopt{-} \hlnum{1}\hlstd{)}  \hlcom{## p-value of the test statistic}
    \hlkwd{list}\hlstd{(}\hlkwc{test} \hlstd{= test,} \hlkwc{critical} \hlstd{= crit,} \hlkwc{p.value} \hlstd{= pvalue,} \hlkwc{df1} \hlstd{= p,} \hlkwc{df2} \hlstd{= n} \hlopt{-}
        \hlstd{p} \hlopt{-} \hlnum{1}\hlstd{)}
\hlstd{\}}


\hlcom{# Subgrup Estadi 1}
\hlstd{x1} \hlkwb{<-} \hlkwd{subset}\hlstd{(dat, estadi} \hlopt{==} \hlnum{1}\hlstd{)[,} \hlnum{1}\hlopt{:}\hlnum{2}\hlstd{]}
\hlcom{# Subgrup Estadi 2}
\hlstd{x2} \hlkwb{<-} \hlkwd{subset}\hlstd{(dat, estadi} \hlopt{==} \hlnum{2}\hlstd{)[,} \hlnum{1}\hlopt{:}\hlnum{2}\hlstd{]}
\hlstd{res} \hlkwb{<-} \hlkwd{hotel2T2}\hlstd{(x1, x2,} \hlkwc{a} \hlstd{=} \hlnum{0.05}\hlstd{)}
\end{alltt}
\end{kframe}
\end{knitrout}


Obtenim:
\\
T$^{2}= 38.76 \leq \frac{(167-1)p}{(167-2)}F_{2,164}$
\\
Aleshores ,
38.76$\geq \frac{(167-1)p}{(167-2)}F_{2,164}$. Per tant, rebutgem la hip?tesis nul?la d'igualtat en les matrius de mitjanes.

Una altra manera de comparar les mitjanes, ?s el an?lisi MANOVA. On la hipotesis nula ?s igualtat en la matriu de mitjanes dels diferents grups.
\\
Utilitzarem la t?cnica de Wilk's. On l'estad?stic ?s:
\\
$\Lambda *=\frac{\left | W \right |}{\left | B+W \right |}$
\\
\right |
\begin{knitrout}
\definecolor{shadecolor}{rgb}{0.969, 0.969, 0.969}\color{fgcolor}\begin{kframe}
\begin{alltt}
\hlstd{groupF} \hlkwb{<-} \hlkwd{factor}\hlstd{(dat[,} \hlnum{3}\hlstd{])}
\hlstd{Yvar} \hlkwb{<-} \hlkwd{as.matrix}\hlstd{(dat[,} \hlnum{1}\hlopt{:}\hlnum{2}\hlstd{])}
\hlstd{x} \hlkwb{<-} \hlkwd{summary}\hlstd{(}\hlkwd{manova}\hlstd{(Yvar} \hlopt{~} \hlstd{groupF,} \hlkwc{data} \hlstd{= dat),} \hlkwc{test} \hlstd{=} \hlstr{"Wilks"}\hlstd{)}
\hlstd{x}
\end{alltt}
\begin{verbatim}
##            Df Wilks approx F num Df den Df  Pr(>F)    
## groupF      1 0.679     38.8      2    164 1.6e-14 ***
## Residuals 165                                         
## ---
## Signif. codes:  0 '***' 0.001 '**' 0.01 '*' 0.05 '.' 0.1 ' ' 1
\end{verbatim}
\end{kframe}
\end{knitrout}


En aquest cas tampoc podem assumir igualtat en les matrius de mitjanes.

\subsection{Perform a multivariate comparison of covariance matrices.}
Per compara els dos vectors de mitjanes, primerament utilitzarem el Box's test:
\begin{knitrout}
\definecolor{shadecolor}{rgb}{0.969, 0.969, 0.969}\color{fgcolor}\begin{kframe}
\begin{alltt}
\hlcom{# Box's test}
\hlstd{cov.Mtest} \hlkwb{=} \hlkwa{function}\hlstd{(}\hlkwc{x}\hlstd{,} \hlkwc{ina}\hlstd{,} \hlkwc{a} \hlstd{=} \hlnum{0.05}\hlstd{) \{}
    \hlstd{p} \hlkwb{=} \hlkwd{ncol}\hlstd{(x)}  \hlcom{## dimension of the data set}
    \hlstd{n} \hlkwb{=} \hlkwd{nrow}\hlstd{(x)}  \hlcom{## sample size}
    \hlstd{k} \hlkwb{=} \hlkwd{max}\hlstd{(ina)}  \hlcom{## number of groups}
    \hlstd{nu} \hlkwb{=} \hlkwd{rep}\hlstd{(}\hlnum{0}\hlstd{, k)}  \hlcom{## the sample size of each group will be stored here later}
    \hlstd{pame} \hlkwb{=} \hlkwd{rep}\hlstd{(}\hlnum{0}\hlstd{, k)}  \hlcom{## the determinant of each covariance will be stored here}
    \hlcom{## t calculate the covariance matrix of each group}
    \hlkwa{for} \hlstd{(i} \hlkwa{in} \hlnum{1}\hlopt{:}\hlstd{k) \{}
        \hlstd{nu[i]} \hlkwb{=} \hlkwd{sum}\hlstd{(ina} \hlopt{==} \hlstd{i)}
    \hlstd{\}}
    \hlstd{z} \hlkwb{=} \hlkwd{cbind}\hlstd{(x, ina)}
    \hlstd{mat} \hlkwb{=} \hlkwd{array}\hlstd{(}\hlkwc{dim} \hlstd{=} \hlkwd{c}\hlstd{(p} \hlopt{+} \hlnum{1}\hlstd{, p} \hlopt{+} \hlnum{1}\hlstd{, k))}
    \hlstd{mat1} \hlkwb{=} \hlkwd{array}\hlstd{(}\hlkwc{dim} \hlstd{=} \hlkwd{c}\hlstd{(p, p, k))}

    \hlkwa{for} \hlstd{(i} \hlkwa{in} \hlnum{1}\hlopt{:}\hlstd{k) \{}
        \hlstd{mat[, , i]} \hlkwb{=} \hlkwd{cov}\hlstd{(z[ina} \hlopt{==} \hlstd{i, ])}
    \hlstd{\}}
    \hlstd{mat} \hlkwb{=} \hlstd{mat[}\hlnum{1}\hlopt{:}\hlstd{p,} \hlnum{1}\hlopt{:}\hlstd{p,} \hlnum{1}\hlopt{:}\hlstd{k]}
    \hlkwa{for} \hlstd{(i} \hlkwa{in} \hlnum{1}\hlopt{:}\hlstd{k) \{}
        \hlstd{mat1[, , i]} \hlkwb{=} \hlstd{mat[, , i]} \hlopt{*} \hlstd{nu[i]}
    \hlstd{\}}

    \hlcom{## calculate the pooled covariance matrix}
    \hlstd{Sp} \hlkwb{=} \hlkwd{apply}\hlstd{(mat1,} \hlnum{1}\hlopt{:}\hlnum{2}\hlstd{, sum)}
    \hlstd{Sp} \hlkwb{=} \hlstd{Sp}\hlopt{/}\hlstd{(n} \hlopt{-} \hlstd{k)}
    \hlkwa{for} \hlstd{(i} \hlkwa{in} \hlnum{1}\hlopt{:}\hlstd{k) \{}
        \hlcom{## this 'for' function calculates the determinant of each covariance matrix}
        \hlstd{pame[i]} \hlkwb{=} \hlkwd{det}\hlstd{((nu[i]}\hlopt{/}\hlstd{(nu[i]} \hlopt{-} \hlnum{1}\hlstd{))} \hlopt{*} \hlstd{mat[, , i])}
    \hlstd{\}}
    \hlstd{pamela} \hlkwb{=} \hlkwd{det}\hlstd{(Sp)}  \hlcom{## determinant of the pooled covariance matrix}

    \hlcom{## construct the test statistic}
    \hlstd{test1} \hlkwb{=} \hlkwd{log}\hlstd{(pamela}\hlopt{/}\hlstd{pame)}
    \hlstd{test2} \hlkwb{=} \hlkwd{sum}\hlstd{((nu} \hlopt{-} \hlnum{1}\hlstd{)} \hlopt{*} \hlstd{test1)}
    \hlstd{gama1} \hlkwb{=} \hlstd{(}\hlnum{2} \hlopt{*} \hlstd{(p}\hlopt{^}\hlnum{2}\hlstd{)} \hlopt{+} \hlnum{3} \hlopt{*} \hlstd{p} \hlopt{-} \hlnum{1}\hlstd{)}\hlopt{/}\hlstd{(}\hlnum{6} \hlopt{*} \hlstd{(p} \hlopt{+} \hlnum{1}\hlstd{)} \hlopt{*} \hlstd{(k} \hlopt{-} \hlnum{1}\hlstd{))}
    \hlstd{gama2} \hlkwb{=} \hlstd{gama1} \hlopt{*} \hlstd{(}\hlkwd{sum}\hlstd{(}\hlnum{1}\hlopt{/}\hlstd{(nu} \hlopt{-} \hlnum{1}\hlstd{))} \hlopt{-} \hlnum{1}\hlopt{/}\hlstd{(n} \hlopt{-} \hlstd{k))}
    \hlstd{gama} \hlkwb{=} \hlnum{1} \hlopt{-} \hlstd{gama1} \hlopt{*} \hlstd{gama2}
    \hlstd{test} \hlkwb{=} \hlstd{gama} \hlopt{*} \hlstd{test2}  \hlcom{## this is the M}
    \hlstd{df} \hlkwb{=} \hlnum{0.5} \hlopt{*} \hlstd{p} \hlopt{*} \hlstd{(p} \hlopt{+} \hlnum{1}\hlstd{)} \hlopt{*} \hlstd{(k} \hlopt{-} \hlnum{1}\hlstd{)}  \hlcom{## degrees of freedom of the chi-square distribution}
    \hlstd{pvalue} \hlkwb{=} \hlnum{1} \hlopt{-} \hlkwd{pchisq}\hlstd{(test, df)}  \hlcom{## p-value of the test statistic}
    \hlstd{crit} \hlkwb{=} \hlkwd{qchisq}\hlstd{(}\hlnum{1} \hlopt{-} \hlstd{a, df)}  \hlcom{#critical value of the chi-square distribution}
    \hlkwd{list}\hlstd{(}\hlkwc{M.test} \hlstd{= test,} \hlkwc{degrees} \hlstd{= df,} \hlkwc{critical} \hlstd{= crit,} \hlkwc{p.value} \hlstd{= pvalue)}
\hlstd{\}}


\hlstd{res1} \hlkwb{<-} \hlkwd{cov.Mtest}\hlstd{(dat[,} \hlnum{1}\hlopt{:}\hlnum{2}\hlstd{],} \hlkwd{as.numeric}\hlstd{(dat[,} \hlnum{3}\hlstd{]),} \hlkwc{a} \hlstd{=} \hlnum{0.05}\hlstd{)}
\end{alltt}
\end{kframe}
\end{knitrout}

Tamb? utilitzem el test de la m?xima likelihood per comparar les matrius de covari?ncies.
\begin{knitrout}
\definecolor{shadecolor}{rgb}{0.969, 0.969, 0.969}\color{fgcolor}\begin{kframe}
\begin{alltt}
\hlcom{# Test de la m?xima likelihod}
\hlstd{cov.likel} \hlkwb{=} \hlkwa{function}\hlstd{(}\hlkwc{x}\hlstd{,} \hlkwc{ina}\hlstd{,} \hlkwc{a} \hlstd{=} \hlnum{0.05}\hlstd{) \{}
    \hlstd{p} \hlkwb{=} \hlkwd{ncol}\hlstd{(x)}  \hlcom{## dimension of the data set}
    \hlstd{n} \hlkwb{=} \hlkwd{nrow}\hlstd{(x)}  \hlcom{## sample size}
    \hlstd{k} \hlkwb{=} \hlkwd{max}\hlstd{(ina)}  \hlcom{## number of groups}
    \hlstd{nu} \hlkwb{=} \hlkwd{rep}\hlstd{(}\hlnum{0}\hlstd{, k)}  \hlcom{## the sample size of each group will be stored later}
    \hlstd{pame} \hlkwb{=} \hlkwd{rep}\hlstd{(}\hlnum{0}\hlstd{, k)}  \hlcom{## the determinant of each covariance will be stored}

    \hlkwa{for} \hlstd{(i} \hlkwa{in} \hlnum{1}\hlopt{:}\hlstd{k) \{}
        \hlstd{nu[i]} \hlkwb{=} \hlkwd{sum}\hlstd{(ina} \hlopt{==} \hlstd{i)}
    \hlstd{\}}
    \hlstd{z} \hlkwb{=} \hlkwd{cbind}\hlstd{(x, ina)}
    \hlstd{mat} \hlkwb{=} \hlkwd{array}\hlstd{(}\hlkwc{dim} \hlstd{=} \hlkwd{c}\hlstd{(p} \hlopt{+} \hlnum{1}\hlstd{, p} \hlopt{+} \hlnum{1}\hlstd{, k))}
    \hlstd{mat1} \hlkwb{=} \hlkwd{array}\hlstd{(}\hlkwc{dim} \hlstd{=} \hlkwd{c}\hlstd{(p, p, k))}

    \hlkwa{for} \hlstd{(i} \hlkwa{in} \hlnum{1}\hlopt{:}\hlstd{k) \{}
        \hlstd{mat[, , i]} \hlkwb{=} \hlkwd{cov}\hlstd{(z[ina} \hlopt{==} \hlstd{i, ])}
    \hlstd{\}}
    \hlstd{mat} \hlkwb{=} \hlstd{mat[}\hlnum{1}\hlopt{:}\hlstd{p,} \hlnum{1}\hlopt{:}\hlstd{p,} \hlnum{1}\hlopt{:}\hlstd{k]}

    \hlcom{## create the pooled covariance matrix}
    \hlkwa{for} \hlstd{(i} \hlkwa{in} \hlnum{1}\hlopt{:}\hlstd{k) \{}
        \hlstd{mat1[, , i]} \hlkwb{=} \hlstd{mat[, , i]} \hlopt{*} \hlstd{nu[i]}
    \hlstd{\}}
    \hlstd{Sp} \hlkwb{=} \hlkwd{apply}\hlstd{(mat1,} \hlnum{1}\hlopt{:}\hlnum{2}\hlstd{, sum)}
    \hlstd{Sp} \hlkwb{=} \hlstd{Sp}\hlopt{/}\hlstd{n}

    \hlcom{## calculate the determinant of each covariance matrix}
    \hlkwa{for} \hlstd{(i} \hlkwa{in} \hlnum{1}\hlopt{:}\hlstd{k) \{}
        \hlstd{pame[i]} \hlkwb{=} \hlkwd{det}\hlstd{(mat[, , i])}
    \hlstd{\}}
    \hlstd{pamela} \hlkwb{=} \hlkwd{det}\hlstd{(Sp)}  \hlcom{## determinant of the pooled covariance matrix}

    \hlstd{test1} \hlkwb{=} \hlkwd{log}\hlstd{(pamela}\hlopt{/}\hlstd{pame)}  \hlcom{## divides the determinant of the pooled covariance}
    \hlcom{## matrix with every covariance matrix}
    \hlstd{test} \hlkwb{=} \hlkwd{sum}\hlstd{(nu} \hlopt{*} \hlstd{test1)}  \hlcom{## test statistic}
    \hlstd{df} \hlkwb{=} \hlnum{0.5} \hlopt{*} \hlstd{p} \hlopt{*} \hlstd{(p} \hlopt{+} \hlnum{1}\hlstd{)} \hlopt{*} \hlstd{(k} \hlopt{-} \hlnum{1}\hlstd{)}  \hlcom{## degrees of freedom of the asymptotic chi-square}
    \hlstd{pvalue} \hlkwb{=} \hlnum{1} \hlopt{-} \hlkwd{pchisq}\hlstd{(test, df)}  \hlcom{## p-value of the test statistic}
    \hlstd{crit} \hlkwb{=} \hlkwd{qchisq}\hlstd{(}\hlnum{1} \hlopt{-} \hlstd{a, df)}  \hlcom{#critical value of the chi-square distribution}
    \hlkwd{list}\hlstd{(}\hlkwc{test} \hlstd{= test,} \hlkwc{degrees} \hlstd{= df,} \hlkwc{critical} \hlstd{= crit,} \hlkwc{p.value} \hlstd{= pvalue)}
\hlstd{\}}


\hlstd{res2} \hlkwb{<-} \hlkwd{cov.likel}\hlstd{(dat[,} \hlnum{1}\hlopt{:}\hlnum{2}\hlstd{],} \hlkwd{as.numeric}\hlstd{(dat[,} \hlnum{3}\hlstd{]),} \hlkwc{a} \hlstd{=} \hlnum{0.05}\hlstd{)}
\end{alltt}
\end{kframe}
\end{knitrout}


% latex table generated in R 3.0.1 by xtable 1.7-1 package
% Wed Jun 25 14:26:42 2014
\begin{table}[ht]
\centering
\caption{Igualtat en la matriu de cov?riancies} 
\begin{tabular}{rrrrr}
  \hline
 & Estad?stic & G.ll. & Valor Cr?tic & P-Valor \\ 
  \hline
Box's test & 26.338 & 3.000 & 7.815 & 0.000 \\ 
  Test de la m?xima likelihood & 26.781 & 3.000 & 7.815 & 0.000 \\ 
   \hline
\end{tabular}
\end{table}




\subsection{Construct the Fisher's linear discriminant function and the quadratic discriminant function using your own functions}
\subsubsection{Funci? lineal de Fisher}

Per tal de poder realitzar la funci? lineal de Fisher, calculem S$_{pooled}$ amb la f?rmula:
\begin{center}
S$_{pooled}$=$\frac{(n_{1}-1)S_{1}+(n_{2}-1)S_{2}}{n-2}$
\end{center}
\begin{knitrout}
\definecolor{shadecolor}{rgb}{0.969, 0.969, 0.969}\color{fgcolor}\begin{kframe}
\begin{alltt}
\hlcom{# Estadi 1}
\hlstd{x1} \hlkwb{<-} \hlkwd{subset}\hlstd{(dat, estadi} \hlopt{==} \hlnum{1}\hlstd{)[,} \hlnum{1}\hlopt{:}\hlnum{2}\hlstd{]}
\hlcom{# Estadi 2}
\hlstd{x2} \hlkwb{<-} \hlkwd{subset}\hlstd{(dat, estadi} \hlopt{==} \hlnum{2}\hlstd{)[,} \hlnum{1}\hlopt{:}\hlnum{2}\hlstd{]}
\hlcom{# Dimensi? de les dades}
\hlstd{p} \hlkwb{=} \hlkwd{ncol}\hlstd{(x1)}
\hlcom{# Mostra en el estadi 1}
\hlstd{n1} \hlkwb{=} \hlkwd{nrow}\hlstd{(x1)}
\hlcom{# Mostra en el estadi 2}
\hlstd{n2} \hlkwb{=} \hlkwd{nrow}\hlstd{(x2)}
\hlcom{# Total mostra}
\hlstd{n} \hlkwb{=} \hlstd{n1} \hlopt{+} \hlstd{n2}
\hlcom{# Vector mitjana mostral estadi 1}
\hlstd{xbar1} \hlkwb{=} \hlkwd{as.matrix}\hlstd{(}\hlkwd{rbind}\hlstd{(}\hlkwd{apply}\hlstd{(x1,} \hlnum{2}\hlstd{, mean)))}  \hlcom{#}
\hlcom{# Vector mitjana mostral estadi 2}
\hlstd{xbar2} \hlkwb{=} \hlkwd{as.matrix}\hlstd{(}\hlkwd{rbind}\hlstd{(}\hlkwd{apply}\hlstd{(x2,} \hlnum{2}\hlstd{, mean)))}
\hlcom{# Diferencia vector de mitjanes}
\hlstd{dbar} \hlkwb{=} \hlstd{xbar1} \hlopt{-} \hlstd{xbar2}
\hlcom{## pooled covariance matrix}
\hlstd{v} \hlkwb{=} \hlstd{((n1} \hlopt{-} \hlnum{1}\hlstd{)} \hlopt{*} \hlkwd{var}\hlstd{(x1)} \hlopt{+} \hlstd{(n2} \hlopt{-} \hlnum{1}\hlstd{)} \hlopt{*} \hlkwd{var}\hlstd{(x2))}\hlopt{/}\hlstd{(n} \hlopt{-} \hlnum{2}\hlstd{)}
\end{alltt}
\end{kframe}
\end{knitrout}

Obtenim:

\begin{center}

$\hat{S}_{pooled}$=$\begin{pmatrix}
301.4 & 31.02\\ 
31.02 & 222.52
\end{pmatrix}$
\end{center}
\\
\\
Per obtenir la funci? discriminant de fisher calculem:
\begin{center}
 Y=($\mu_{1}-\mu_{1}$)'S$_{pooled}^{-1}$x
\end{center}
\begin{knitrout}
\definecolor{shadecolor}{rgb}{0.969, 0.969, 0.969}\color{fgcolor}\begin{kframe}
\begin{alltt}
\hlcom{# Inversa de Spooled}
\hlstd{v_inverse} \hlkwb{<-} \hlkwd{solve}\hlstd{(v)}
\hlcom{# Funci? discriminant de fisher}
\hlstd{y} \hlkwb{=} \hlstd{dbar} \hlopt \hlstd{v_inverse}
\end{alltt}
\end{kframe}
\end{knitrout}

El resultat d'aquest ?s:
\begin{pmatrix}
-22.1439 &-9.7782
\end{pmatrix}
\begin{pmatrix}
0.0034 & -0.0005 \\ 
 -0.00054  &0.0046
\end{pmatrix}
\begin{pmatrix}
X$_{1}$\\ X$_{2}$

\end{pmatrix}=
\\
=-0.06995286X$_{1}$ -0.03418953X$_{2}$

Per tal d 'assignar les noves observacions, apliquem la regla:
\begin{itemize}
\item X$_{0}$ en el estadi 1 si : ($\mu_2 -\mu_1$)' $\Sigma^{-1}X_{0} -m\leq$0 
\item X$_{0}$ en el estadi 2 si : ($\mu_2 -\mu_1$)' $\Sigma^{-1}X_{0} -m\ge$0  
\end{itemize}
on,
\\
$\tilde{m}$=($\mu_{1}-\mu_{1}$)'S$_{pooled}^{-1}$($\mu_{1}+\mu_{1}$)
\begin{knitrout}
\definecolor{shadecolor}{rgb}{0.969, 0.969, 0.969}\color{fgcolor}\begin{kframe}
\begin{alltt}
\hlstd{lda_fisher} \hlkwb{<-} \hlkwa{function}\hlstd{(}\hlkwc{xbar1}\hlstd{,} \hlkwc{xbar2}\hlstd{,} \hlkwc{v_inverse}\hlstd{,} \hlkwc{newdat}\hlstd{) \{}
    \hlcom{# Calcula la diferencia dels vectors de mitjanes}
    \hlstd{dbar} \hlkwb{<-} \hlstd{xbar2} \hlopt{-} \hlstd{xbar1}
    \hlcom{# Calcula fisher}
    \hlstd{y} \hlkwb{=} \hlstd{dbar} \hlopt \hlstd{v_inverse}
    \hlstd{m} \hlkwb{<-} \hlstd{(dbar} \hlopt \hlstd{v_inverse} \hlopt \hlkwd{t}\hlstd{(xbar2} \hlopt{+} \hlstd{xbar1))}\hlopt{/}\hlnum{2}


    \hlstd{d} \hlkwb{<-} \hlstd{y} \hlopt \hlkwd{t}\hlstd{(newdat)}
    \hlstd{xo} \hlkwb{<-} \hlkwd{ifelse}\hlstd{(d} \hlopt{-} \hlstd{m} \hlopt{>=} \hlnum{0}\hlstd{,} \hlnum{2}\hlstd{,} \hlkwd{ifelse}\hlstd{(d} \hlopt{-} \hlstd{m} \hlopt{<} \hlnum{0}\hlstd{,} \hlnum{1}\hlstd{,} \hlnum{0}\hlstd{))}
    \hlstd{x} \hlkwb{<-} \hlkwd{c}\hlstd{(xo, d)}

    \hlkwd{return}\hlstd{(x)}
\hlstd{\}}
\hlstd{lda} \hlkwb{<-} \hlkwa{NULL}
\hlkwa{for} \hlstd{(i} \hlkwa{in} \hlnum{1}\hlopt{:}\hlkwd{nrow}\hlstd{(dat)) \{}
    \hlstd{lda} \hlkwb{<-} \hlkwd{rbind}\hlstd{(lda,} \hlkwd{lda_fisher}\hlstd{(xbar1, xbar2, v_inverse,} \hlkwd{as.matrix}\hlstd{(dat[,} \hlnum{1}\hlopt{:}\hlnum{2}\hlstd{][i,}
        \hlstd{])))}
\hlstd{\}}
\end{alltt}
\end{kframe}
\end{knitrout}

\begin{knitrout}
\definecolor{shadecolor}{rgb}{0.969, 0.969, 0.969}\color{fgcolor}\begin{kframe}
\begin{alltt}
\hlstd{p1} \hlkwb{<-} \hlkwd{nrow}\hlstd{(x1)}\hlopt{/}\hlkwd{nrow}\hlstd{(dat)}
\hlstd{p2} \hlkwb{<-} \hlkwd{nrow}\hlstd{(x2)}\hlopt{/}\hlkwd{nrow}\hlstd{(dat)}
\end{alltt}
\end{kframe}
\end{knitrout}


En aquest cas no em tingut en compte les probabilitats a priori, que en aquest cas s?n diferents pels dos grups.Segons el llibre {\bf{Applied Multivariate Statistical Analysis}} de Johnson utilitzarem la seg?ent regla per assignar noves observacions en els estadis:
\begin{itemize}
\item p$_{1}$=\frac{n_{1}}{n}$ = 0.4551
\item p$_{2}$=\frac{n_{1}}{n}$ = 0.5449
\end{itemize}


\begin{knitrout}
\definecolor{shadecolor}{rgb}{0.969, 0.969, 0.969}\color{fgcolor}\begin{kframe}
\begin{alltt}
\hlstd{p1} \hlkwb{<-} \hlkwd{nrow}\hlstd{(x1)}\hlopt{/}\hlkwd{nrow}\hlstd{(dat)}
\hlstd{p2} \hlkwb{<-} \hlkwd{nrow}\hlstd{(x2)}\hlopt{/}\hlkwd{nrow}\hlstd{(dat)}
\hlstd{lda_fisher_priori} \hlkwb{<-} \hlkwa{function}\hlstd{(}\hlkwc{xbar1}\hlstd{,} \hlkwc{xbar2}\hlstd{,} \hlkwc{v_inverse}\hlstd{,} \hlkwc{newdat}\hlstd{,} \hlkwc{p1}\hlstd{,} \hlkwc{p2}\hlstd{) \{}
    \hlcom{# Calcula la diferencia dels vectors de mitjanes}
    \hlstd{dbar} \hlkwb{<-} \hlstd{xbar2} \hlopt{-} \hlstd{xbar1}
    \hlcom{# Calcula fisher}
    \hlstd{y} \hlkwb{=} \hlstd{dbar} \hlopt \hlstd{v_inverse}
    \hlstd{m} \hlkwb{<-} \hlkwd{log}\hlstd{(p1}\hlopt{/}\hlstd{p2)}


    \hlstd{d} \hlkwb{<-} \hlstd{y} \hlopt \hlkwd{t}\hlstd{(newdat)}
    \hlstd{xo} \hlkwb{<-} \hlkwd{ifelse}\hlstd{(d} \hlopt{-} \hlstd{m} \hlopt{>=} \hlnum{0}\hlstd{,} \hlnum{1}\hlstd{,} \hlkwd{ifelse}\hlstd{(d} \hlopt{-} \hlstd{m} \hlopt{<} \hlnum{0}\hlstd{,} \hlnum{2}\hlstd{,} \hlnum{0}\hlstd{))}

    \hlstd{x} \hlkwb{<-} \hlkwd{c}\hlstd{(xo, d)}

    \hlkwd{return}\hlstd{(x)}
\hlstd{\}}
\hlstd{lda_priori} \hlkwb{<-} \hlkwa{NULL}
\hlkwa{for} \hlstd{(i} \hlkwa{in} \hlnum{1}\hlopt{:}\hlkwd{nrow}\hlstd{(dat)) \{}
    \hlstd{lda_priori} \hlkwb{<-} \hlkwd{rbind}\hlstd{(lda_priori,} \hlkwd{lda_fisher_priori}\hlstd{(xbar1, xbar2, v_inverse,}
        \hlkwd{as.matrix}\hlstd{(dat[,} \hlnum{1}\hlopt{:}\hlnum{2}\hlstd{][i, ]), p1, p2))}
\hlstd{\}}
\end{alltt}
\end{kframe}
\end{knitrout}


\subsubsection{Funci? quad?tica discriminant}
Per tal de portar a terme la funci? discriminant quadr?tica utilitzarem la regla:
\\
$\frac{1}{2}x_{0}'(S_{1}^{-1}-S_{2}^{-1})x_0}+(\hat{x}_{1}'S_{1}^{-1}-{\hat{x}_{2}'S_{2}^{-1})x_{0}-k\geq log(\frac{p_1}{p{2}})$
\\
on,
\\
k=$\frac{1}{2}\left ( \frac{\left | S_{1} \right |}{\left | S_{2} \right |} \right )+\frac{1}{2}(\hat{x}_{1}'S_{1}^{-1}\hat{x}_{1}-\hat{x}_{2}'S_{2}^{-1}\hat{x}_{2})$
\begin{knitrout}
\definecolor{shadecolor}{rgb}{0.969, 0.969, 0.969}\color{fgcolor}\begin{kframe}
\begin{alltt}
\hlcom{# S1}
\hlstd{s1} \hlkwb{<-} \hlkwd{cov}\hlstd{(x1)}
\hlcom{# s2}
\hlstd{s2} \hlkwb{<-} \hlkwd{cov}\hlstd{(x2)}
\hlstd{p1} \hlkwb{<-} \hlkwd{nrow}\hlstd{(x1)}\hlopt{/}\hlkwd{nrow}\hlstd{(dat)}
\hlstd{p2} \hlkwb{<-} \hlkwd{nrow}\hlstd{(x2)}\hlopt{/}\hlkwd{nrow}\hlstd{(dat)}
\hlstd{k} \hlkwb{<-} \hlnum{1}\hlopt{/}\hlnum{2} \hlopt{*} \hlkwd{log}\hlstd{(}\hlkwd{det}\hlstd{(s1)}\hlopt{/}\hlkwd{det}\hlstd{(s2))} \hlopt{+} \hlnum{1}\hlopt{/}\hlnum{2} \hlopt{*} \hlstd{((xbar1)} \hlopt \hlkwd{solve}\hlstd{(s1)} \hlopt \hlkwd{t}\hlstd{(xbar1)} \hlopt{-}
    \hlstd{xbar2} \hlopt \hlkwd{solve}\hlstd{(s2)} \hlopt \hlkwd{t}\hlstd{(xbar2))}
\hlstd{qda} \hlkwb{<-} \hlkwa{NULL}
\hlstd{X} \hlkwb{<-} \hlkwd{as.matrix}\hlstd{(dat[,} \hlnum{1}\hlopt{:}\hlnum{2}\hlstd{])}
\hlkwa{for} \hlstd{(i} \hlkwa{in} \hlnum{1}\hlopt{:}\hlkwd{nrow}\hlstd{(dat)) \{}
    \hlstd{yq} \hlkwb{<-} \hlopt{-}\hlnum{1}\hlopt{/}\hlnum{2} \hlopt{*} \hlkwd{t}\hlstd{(X[i, ])} \hlopt \hlstd{(}\hlkwd{solve}\hlstd{(s1)} \hlopt{-} \hlkwd{solve}\hlstd{(s2))} \hlopt \hlstd{X[i, ]} \hlopt{+} \hlstd{(xbar1} \hlopt
        \hlkwd{solve}\hlstd{(s1)} \hlopt{-} \hlstd{xbar2} \hlopt \hlkwd{solve}\hlstd{(s2))} \hlopt \hlstd{X[i, ]}
    \hlstd{xo} \hlkwb{<-} \hlkwd{ifelse}\hlstd{(yq} \hlopt{-} \hlstd{k} \hlopt{>=} \hlkwd{log}\hlstd{(p1}\hlopt{/}\hlstd{p2),} \hlnum{1}\hlstd{,} \hlkwd{ifelse}\hlstd{(yq} \hlopt{-} \hlstd{k} \hlopt{<} \hlkwd{log}\hlstd{(p1}\hlopt{/}\hlstd{p2),} \hlnum{2}\hlstd{,} \hlnum{NA}\hlstd{))}
    \hlstd{qda} \hlkwb{<-} \hlkwd{rbind}\hlstd{(qda,} \hlkwd{cbind}\hlstd{(yq, xo))}
\hlstd{\}}
\hlstd{res} \hlkwb{<-} \hlkwd{cbind}\hlstd{(dat, qda[,} \hlnum{2}\hlstd{])}
\hlkwd{table}\hlstd{(res[,} \hlnum{3}\hlstd{], res[,} \hlnum{4}\hlstd{])}
\end{alltt}
\end{kframe}
\end{knitrout}


\subsection{Show the derived discriminant functions on a scatter plot of the original data.}
\begin{knitrout}
\definecolor{shadecolor}{rgb}{0.969, 0.969, 0.969}\color{fgcolor}\begin{kframe}
\begin{alltt}
\hlkwd{jpeg}\hlstd{(}\hlstr{"./fig1.jpg"}\hlstd{)}
\hlstd{x} \hlkwb{<-} \hlkwd{cbind}\hlstd{(lda, dat[,} \hlnum{3}\hlstd{])}

\hlkwd{plot}\hlstd{(x[,} \hlnum{2}\hlstd{],} \hlkwc{bty} \hlstd{=} \hlstr{"n"}\hlstd{,} \hlkwc{type} \hlstd{=} \hlstr{"n"}\hlstd{,} \hlkwc{main} \hlstd{=} \hlstr{"Funci? discriminant de Fisher"}\hlstd{,}
    \hlkwc{ylab} \hlstd{=} \hlstr{"first linear discriminant"}\hlstd{,} \hlkwc{xlab} \hlstd{=} \hlstr{""}\hlstd{)}
\hlkwd{points}\hlstd{(}\hlkwd{subset}\hlstd{(x[,} \hlnum{2}\hlstd{], x[,} \hlnum{3}\hlstd{]} \hlopt{==} \hlnum{1}\hlstd{),} \hlkwc{col} \hlstd{=} \hlstr{"orangered"}\hlstd{,} \hlkwc{pch} \hlstd{=} \hlnum{20}\hlstd{)}
\hlkwd{points}\hlstd{(}\hlkwd{subset}\hlstd{(x[,} \hlnum{2}\hlstd{], x[,} \hlnum{3}\hlstd{]} \hlopt{==} \hlnum{2}\hlstd{),} \hlkwc{col} \hlstd{=} \hlstr{"blue"}\hlstd{,} \hlkwc{pch} \hlstd{=} \hlnum{20}\hlstd{)}
\hlkwd{legend}\hlstd{(}\hlstr{"bottomright"}\hlstd{,} \hlkwd{c}\hlstd{(}\hlstr{"Estadi 1"}\hlstd{,} \hlstr{"Estadi 2"}\hlstd{),} \hlkwc{col} \hlstd{=} \hlkwd{c}\hlstd{(}\hlstr{"orangered"}\hlstd{,} \hlstr{"blue"}\hlstd{),}
    \hlkwc{pch} \hlstd{=} \hlnum{20}\hlstd{,} \hlkwc{bty} \hlstd{=} \hlstr{"n"}\hlstd{)}

\hlkwd{dev.off}\hlstd{()}
\end{alltt}
\end{kframe}
\end{knitrout}

No ?s pot realitzar el gr?fic de la funci? discriminant quadr?tica, ja que es basa en un score/ regla d'assignaci?.
\begin{figure}[H]
\begin{center}
\includegraphics[width=10cm]{./fig1.jpg}
\end{center}
\end{figure}

\subsection{Estimate the misclassification rate.}
\subsubsection{Funci? lineal de Fisher}
No tinguen en compte les probabilitats a priori:
\begin{kframe}
\begin{alltt}
\hlstd{ct} \hlkwb{<-} \hlkwd{table}\hlstd{(dat}\hlopt{$}\hlstd{estadi, lda[,} \hlnum{1}\hlstd{])}
\hlkwd{colnames}\hlstd{(ct)} \hlkwb{<-} \hlkwd{c}\hlstd{(}\hlstr{"Allocated to Estadi 1"}\hlstd{,} \hlstr{"Allocated to Estadi 2"}\hlstd{)}
\hlkwd{rownames}\hlstd{(ct)} \hlkwb{<-} \hlkwd{c}\hlstd{(}\hlstr{"Is  Estadi 1"}\hlstd{,} \hlstr{"Is Estadi 2"}\hlstd{)}
\hlstd{x} \hlkwb{<-} \hlstd{ct[}\hlnum{1}\hlstd{,} \hlnum{2}\hlstd{]} \hlopt{+} \hlstd{ct[}\hlnum{2}\hlstd{,} \hlnum{1}\hlstd{]}  \hlcom{#Misclassified}
\hlstd{AR} \hlkwb{<-} \hlkwd{xtable}\hlstd{(ct,} \hlkwc{caption} \hlstd{=} \hlstr{"Misclassification"}\hlstd{)}
\hlkwd{print}\hlstd{(AR,} \hlkwc{sanitize.text.function} \hlstd{=} \hlkwa{function}\hlstd{(}\hlkwc{x}\hlstd{) \{}
    \hlstd{x}
\hlstd{\},} \hlkwc{caption.placement} \hlstd{=} \hlstr{"top"}\hlstd{,} \hlkwc{include.rownames} \hlstd{=} \hlnum{TRUE}\hlstd{)}
\end{alltt}
\end{kframe}% latex table generated in R 3.0.1 by xtable 1.7-1 package
% Wed Jun 25 14:26:43 2014
\begin{table}[ht]
\centering
\caption{Misclassification} 
\begin{tabular}{rrr}
  \hline
 & Allocated to Estadi 1 & Allocated to Estadi 2 \\ 
  \hline
Is  Estadi 1 &  61 &  15 \\ 
  Is Estadi 2 &  21 &  70 \\ 
   \hline
\end{tabular}
\end{table}


Tenim un total de  36  individus mal classificats
\\
Que correspon a una proporci? del mal classificats de: 0.22
\\
\\
\newpage
Tinguen en compte les probabilitats a priori:
\begin{kframe}
\begin{alltt}
\hlstd{ct} \hlkwb{<-} \hlkwd{table}\hlstd{(dat}\hlopt{$}\hlstd{estadi, lda_priori[,} \hlnum{1}\hlstd{])}
\hlkwd{colnames}\hlstd{(ct)} \hlkwb{<-} \hlkwd{c}\hlstd{(}\hlstr{"Allocated to Estadi 1"}\hlstd{)}
\hlkwd{rownames}\hlstd{(ct)} \hlkwb{<-} \hlkwd{c}\hlstd{(}\hlstr{"Is  Estadi 1"}\hlstd{,} \hlstr{"Is Estadi 2"}\hlstd{)}
\hlstd{x} \hlkwb{<-} \hlnum{0}  \hlcom{#Misclassified}
\hlstd{AR} \hlkwb{<-} \hlkwd{xtable}\hlstd{(ct,} \hlkwc{caption} \hlstd{=} \hlstr{"Misclassification"}\hlstd{)}
\hlkwd{print}\hlstd{(AR,} \hlkwc{sanitize.text.function} \hlstd{=} \hlkwa{function}\hlstd{(}\hlkwc{x}\hlstd{) \{}
    \hlstd{x}
\hlstd{\},} \hlkwc{caption.placement} \hlstd{=} \hlstr{"top"}\hlstd{,} \hlkwc{include.rownames} \hlstd{=} \hlnum{TRUE}\hlstd{)}
\end{alltt}
\end{kframe}% latex table generated in R 3.0.1 by xtable 1.7-1 package
% Wed Jun 25 14:26:43 2014
\begin{table}[ht]
\centering
\caption{Misclassification} 
\begin{tabular}{rr}
  \hline
 & Allocated to Estadi 1 \\ 
  \hline
Is  Estadi 1 &  76 \\ 
  Is Estadi 2 &  91 \\ 
   \hline
\end{tabular}
\end{table}


Tenim un total de  0  individus mal classificats
\\
Que correspon a una proporci? del mal classificats de: 0
\\
Utilitzant les probabilitat a priori, trobem que no classifiquem malament cap individu.

\subsubsection{Funci? quad?tica discriminant}

\begin{kframe}
\begin{alltt}
\hlstd{ct} \hlkwb{<-} \hlkwd{table}\hlstd{(dat}\hlopt{$}\hlstd{estadi, qda[,} \hlnum{2}\hlstd{])}
\hlkwd{colnames}\hlstd{(ct)} \hlkwb{<-} \hlkwd{c}\hlstd{(}\hlstr{"Allocated to Estadi 1"}\hlstd{,} \hlstr{"Allocated to Estadi 2"}\hlstd{)}
\hlkwd{rownames}\hlstd{(ct)} \hlkwb{<-} \hlkwd{c}\hlstd{(}\hlstr{"Is  Estadi 1"}\hlstd{,} \hlstr{"Is Estadi 2"}\hlstd{)}
\hlstd{x} \hlkwb{<-} \hlstd{ct[}\hlnum{1}\hlstd{,} \hlnum{2}\hlstd{]} \hlopt{+} \hlstd{ct[}\hlnum{2}\hlstd{,} \hlnum{1}\hlstd{]}  \hlcom{#Misclassified}
\hlstd{AR} \hlkwb{<-} \hlkwd{xtable}\hlstd{(ct,} \hlkwc{caption} \hlstd{=} \hlstr{"Misclassification"}\hlstd{)}
\hlkwd{print}\hlstd{(AR,} \hlkwc{sanitize.text.function} \hlstd{=} \hlkwa{function}\hlstd{(}\hlkwc{x}\hlstd{) \{}
    \hlstd{x}
\hlstd{\},} \hlkwc{caption.placement} \hlstd{=} \hlstr{"top"}\hlstd{,} \hlkwc{include.rownames} \hlstd{=} \hlnum{TRUE}\hlstd{)}
\end{alltt}
\end{kframe}% latex table generated in R 3.0.1 by xtable 1.7-1 package
% Wed Jun 25 14:26:43 2014
\begin{table}[ht]
\centering
\caption{Misclassification} 
\begin{tabular}{rrr}
  \hline
 & Allocated to Estadi 1 & Allocated to Estadi 2 \\ 
  \hline
Is  Estadi 1 &  64 &  12 \\ 
  Is Estadi 2 &  18 &  73 \\ 
   \hline
\end{tabular}
\end{table}


Tenim un total de  30  individus mal classificats
\\
Que correspon a una proporci? del mal classificats de: 0.18

\end{document}
